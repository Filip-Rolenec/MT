%% LyX 2.1.4 created this file.  For more info, see http://www.lyx.org/.
%% Do not edit unless you really know what you are doing.
\documentclass[11pt,czech,american]{book}
\usepackage[T1]{fontenc}
\usepackage[utf8]{inputenc}
\usepackage[a4paper]{geometry}
\geometry{verbose,tmargin=4cm,bmargin=3cm,lmargin=3cm,rmargin=2cm,headheight=0.8cm,headsep=1cm,footskip=0.5cm}
\pagestyle{headings}
\setcounter{secnumdepth}{3}
\usepackage{url}
\usepackage{amsmath}
\usepackage{amsthm}
\usepackage{amssymb}
\usepackage{graphicx}
\usepackage{setspace}
\usepackage[final]{pdfpages}
\usepackage{natbib}
\usepackage{mathrsfs}
\usepackage{algorithm}
\usepackage{algorithmicx}
\usepackage[noend]{algpseudocode}
\usepackage{caption}


\usepackage{array}
\usepackage{ragged2e}

\usepackage{lipsum}
\usepackage{psvectorian}

\DeclareMathOperator*{\argmax}{arg\,max}

\makeatletter
%%%%%%%%%%%%%%%%%%%%%%%%%%%%%% Textclass specific LaTeX commands.
\newenvironment{lyxlist}[1]
{\begin{list}{}
{\settowidth{\labelwidth}{#1}
 \setlength{\leftmargin}{\labelwidth}
 \addtolength{\leftmargin}{\labelsep}
 \renewcommand{\makelabel}[1]{##1\hfil}}}
{\end{list}}

%%%%%%%%%%%%%%%%%%%%%%%%%%%%%% User specified LaTeX commands.
%% Font setup: please leave the LyX font settings all set to 'default'
%% if you want to use any of these packages:

%% Use Times New Roman font for text and Belleek font for math
%% Please make sure that the 'esint' package is turned off in the
%% 'Math options' page.
\usepackage[varg]{txfonts}

%% Use Utopia text with Fourier-GUTenberg math
%\usepackage{fourier}

%% Bitstream Charter text with Math Design math
%\usepackage[charter]{mathdesign}

%%---------------------------------------------------------------------

%% Make the multiline figure/table captions indent so that the second
%% line "hangs" right below the first one.
%\usepackage[format=hang]{caption}

%% Indent even the first paragraph in each section
\usepackage{indentfirst}

%%---------------------------------------------------------------------

%% Disable page numbers in the TOC. LOF, LOT (TOC automatically
%% adds \thispagestyle{chapter} if not overriden
%\addtocontents{toc}{\protect\thispagestyle{empty}}
%\addtocontents{lof}{\protect\thispagestyle{empty}}
%\addtocontents{lot}{\protect\thispagestyle{empty}}

%% Shifts the top line of the TOC (not the title) 1cm upwards 
%% so that the whole TOC fits on 1 page. Additional page size
%% adjustment is performed at the point where the TOC
%% is inserted.
%\addtocontents{toc}{\protect\vspace{-1cm}}

%%---------------------------------------------------------------------

% completely avoid orphans (first lines of a new paragraph on the bottom of a page)
\clubpenalty=9500

% completely avoid widows (last lines of paragraph on a new page)
\widowpenalty=9500

% disable hyphenation of acronyms
\hyphenation{CDFA HARDI HiPPIES IKEM InterTrack MEGIDDO MIMD MPFA DICOM ASCLEPIOS MedInria}

%%---------------------------------------------------------------------

%% Print out all vectors in bold type instead of printing an arrow above them
\renewcommand{\vec}[1]{\boldsymbol{#1}}

% Replace standard \cite by the parenthetical variant \citep
%\renewcommand{\cite}{\citep}

\makeatother

\usepackage{babel}

\newcommand{\ornamentleft}{%
	\psvectorian[width=2em]{2}%
}
\newcommand{\ornamentright}{%
	\psvectorian[width=2em,mirror]{2}%
}

\newcommand{\ornamentheader}[1]{%
	\begin{center}
		\ornamentleft
		\quad{\large\emph{#1}}\quad % style as desired
		\ornamentright
	\end{center}%
}

\newlength{\rlength}\setlength{\rlength}{16cm}
\newcommand{\ruletext}[2][\rlength]{%
	\noindent%
	\parbox{#1}{%
		\noindent\dotfill\raisebox{-.3\ht\strutbox}{#2}\dotfill\par}%
}





\begin{document}
\def\documentdate{July 7, 2017}

\newtheorem{definition}{Definition}[chapter]
\newtheorem{note}{Note}[chapter]
\newtheorem{example}{Example} 
\newtheorem{assumption}{Assumption} 

\newtheorem{theorem}{Theorem}
\newtheorem*{remark}{Remark}

\captionsetup[figure]{labelfont={bf},labelformat={default},labelsep=period,name={Fig.}}


\def\documentdate{\today}

\pagestyle{empty}
{\centering

\noindent %
\begin{minipage}[c]{3cm}%
\noindent \begin{center}
\includegraphics[width=3cm,height=3cm,keepaspectratio]{Images/TITLE/cvut}
\par\end{center}%
\end{minipage}%
\begin{minipage}[c]{0.6\linewidth}%
\begin{center}
\textsc{\large{}Czech Technical University in Prague}{\large{}}\\
{\large{}Faculty of Nuclear Sciences and Physical Engineering}
\par\end{center}%
\end{minipage}%
\begin{minipage}[c]{3cm}%
\noindent \begin{center}
\includegraphics[width=3cm,height=3cm,keepaspectratio]{Images/TITLE/fjfi}
\par\end{center}%
\end{minipage}

\vspace{3cm}


\textbf{\huge{}Real Options Valuation: A Dynamic Programming Approach}{\huge \par}

\vspace{1cm}


\selectlanguage{czech}%
\textbf{\huge{}Oceňování projektů metodou reálných opcí z pohledu dynamického progamování}{\huge \par}

\selectlanguage{american}%
\vspace{2cm}


{\large{}Master's Thesis}{\large \par}

}

\vfill{}

\begin{lyxlist}{MMMMMMMMM}
\begin{singlespace}
\item [{Author:}] \textbf{Filip Rolenec}
\item [{Supervisor:}] \textbf{Ing. Rudolf Kulhavý, DrSc.}
\end{singlespace}

\item [{Language~advisor:}] \textbf{Ing. Rudolf Kulhavý, DrSc.} 
\begin{singlespace}
\item [{Academic~year:}] 2020/2021\end{singlespace}

\end{lyxlist}
\newpage{}

~\newpage{}

~

\vfill{}


\begin{center}
\includepdf[pages={1}]{Images/zadaniMT.pdf}


\par\end{center}

\vfill{}


~\newpage{}

~

\vfill{}


\begin{center}
\includepdf[pages={2}]{Images/zadaniMT.pdf}
\par\end{center}

\vfill{}


~\newpage{}

\noindent \emph{\Large{}Acknowledgment:}{\Large \par}

\noindent I would like to thank my supervisor Ing. Rudolf Kulhavý, DrSc. for his professional guidance and all the advice given while creating this thesis. 

\vfill

\noindent \emph{\Large{}Author's declaration:}{\Large \par}

\noindent I declare that this Master's thesis is entirely
my own work and I have listed all the used sources in the bibliography.

\bigskip{}


\noindent Prague, \documentdate\hfill{}Bc. Filip Rolenec

\vspace{2cm}


\newpage{}

~\newpage{}

\selectlanguage{czech}%
\begin{onehalfspace}
\noindent \emph{Název práce:}

\noindent \textbf{Oceňování projektů metodou reálných opcí z pohledu dynamického progamování}
\end{onehalfspace}

\bigskip{}


\noindent \emph{Autor:} Bc. Filip Rolenec

\bigskip{}


\noindent \emph{Obor:} Matematické inženýrství 


\bigskip{}


\noindent \emph{Druh práce:} Diplomová práce

\bigskip{}


\noindent \emph{Vedoucí práce:} Ing. Rudolf Kulhavý, DrSc.


\bigskip{}


\noindent \emph{Abstrakt:} Investiční příležitosti jsou v současné době oceňovány pomocí řady algoritmů a metrik vzešlých z ekonomické teorie. Nejčastěji používaná metoda \textit{čisté současné hodnoty} (NPV) zohledňuje časovou hodnotu peněz a pro jednoduché projekty dává investorům velmi dobré odhady s minimálními požadavky na matematické znalosti. Složitější projekty, které v této práci chápeme jako projekty s vysokou mírou neurčitosti a existencí následných manažerských rozhodnutí, je možné oceňovat pomocí teorie \textit{reálných opcí} (ROA). Metoda ROA vychází z nedokonalé analogie oceňování finančních opcí a přiznává hodnotu možnostem změny projektového plánu. 

Tato práce má za cíl představit nový rámec pro oceňování investičních příležitostí, jejichž řízení je chápáno  jako stochastický rozhodovací problém. Tento rámec, umožňující využití desítek let výzkumu v oblasti stochastické rozhodovací teorie, pokrývá metody NPV a ROA, přičemž zjemňuje jejich předpoklady. Hlavní přínosy nového oceňovacího rámce jsou: možnost zodhlednění více zdrojů neurčitosti, modelování neurčitosti libovolnou distribucí, přímočaré začlenění bayesovského učení, modelování přístupu k riziku rozhodovacího subjektu a libovolný počet i druh povolených manažerských akcí. 

Nový rámec znatelně rozšiřuje třídu projektů ocenitelných s velkou přesností a lze ho chápat jako sjednocující zobecnění technik oceňování v podnikovém řízení. 


\bigskip{}


\noindent \emph{Klíčová slova:}   Analýza reálných opcí, Blackův-Scholesův model, Čistá současná hodnota, Dynamické programování, Energetika, Oceňování projektů, Stochastická rozhodovací teorie



\selectlanguage{american}%
\vfill{}
~

\begin{onehalfspace}
\noindent \emph{Title:}

\noindent \textbf{Real Options Valuation: A Dynamic Programming Approach}
\end{onehalfspace}

\bigskip{}


\noindent \emph{Author:} Bc. Filip Rolenec

\bigskip{}


\noindent \emph{Abstract:} Investment opportunities are currently valued via metrics and algorithms formed by the economical theory. The majority of investors still values projects with the \textit{net present value} (NPV) method, which takes into account the time value of money and gives solid results for simple projects with minimal requirements on mathematical skills. More complicated projects, which are in this thesis thought of as projects with a substantial degree of inner uncertainty and with an existence of further managerial decisions, can be valued by the \textit{real options analysis} (ROA). This method comes from an imperfect analogy to financial option valuation and it recognizes the value of the ability to change the course of a given project.  

This thesis presents a new valuation framework for projects, which are understood as stochastic decision problems. This framework incorporates the NPV and ROA methods, relaxes their assumptions and allows for decades of research in the field of \textit{stochastic decision theory} (SDT) to be used. The main contributions of the new framework are: ability to incorporate multiple sources of uncertainty, usage of any distribution for uncertainty modeling, ability to conveniently incorporate Bayesian learning, ability to model user's approach to risk and ability to model any type and number of managerial actions.

The new framework significantly expands the class of projects that can be reasonably valued and can be understood as a unification of project valuation in business management.


\bigskip{}


\noindent \emph{Keywords:} Black-Scholes model, Dynamic programming, Net present value, Power industry, Project valuation,  Real option analysis, Stochastic decision control

\newpage{}

~\newpage{}

\pagestyle{plain}

\tableofcontents{}

\newpage{}


\chapter{Introduction}
The ability to systematically and reliably value projects is the core of investment decision making. According to the economical theory presented by \cite{} \footnote{That guy from Duke university}, profits of an investment reflect the value added to the participants of the market, expressed by actual spending of their money. The profits of a venture are understood as the ultimate measure of additional value that has been created. 

A good valuation technique enables companies to increase their profits and as \cite{BerDeM:09} (?)  says: the main and actually only goal of a manager is to increase the wealth of stakeholders. It is thus rather fortunate that, by the logic of \cite{} \footnote{Guy from duke}, in free market, increasing this value coincides with the adding value to all interested participants on the market. One could thus extrapolate and say, that in free market, the goal of a manager is to improve lives of market participants. 

The current state of capital investment valuation techniques is very diverse. Majority of investors rely predominantly on the standard Net Present Value (NPV) technique, its generalization in a form of NPV with scenarios (sometimes called Decision Tree Analysis(DTA)) or NPV-derived metrics with some mostly artificial parameters, such as NPV with Risk Adjusted Discount Rate (RADR) or Internal Rate of Return (IRR). \footnote{Create citations to these statements}

These valuation techniques are usually simple, resulting in a very limited scope of their application. Two main problems are that they do not address the problem of uncertainty or further management ability, in much detail, or in the case of uncertainty the argumentation is misleading \footnote{Risky FCFs should be discounted more...}.

\bigbreak

Many articles and economical books discuss a superior valuation technique, that recognizes the value of further managerial actions and also copes with the uncertainty in more complex way. This technique is called Real Options Analysis (ROA). Project valuation by ROA is used in literature in three forms, differing in the level of its analogy to its origins in valuation of financial options, elegantly handled by the Black-Sholes-Merton (BSM) Nobel prize winning valuation technique \cite{BlaSch73}. 

The value added by recognizing the importance of further management in projects with high degree of uncertainty is significant. Articles \cite{} and \cite{} show, how standard valuation techniques tend to undervalue these projects, leading to potentially tremendous unrealized profits. 

The core of BSM model, and thus its analogy in the form of ROA, is only an argument about the expected value of the underlying asset's price, which is expected to follow log-normally distributed random process. This argument is used in creation of so called risk-neutral probabilities of price movements of the price. 

By reading the publications about ROA one starts to be accustomed to the structure of projects as a collection of managerial decisions. This narrative rings a strong bell to a scholar that has encountered stochastic decision theory (SDT). 

\bigbreak

Stochastic decision theory, with its decades of research, offers a brilliant framework for any decision making problem, which projects effectively are. The analogy for project value has also a potential to be found in the value functions which represent the expected cumulative rewards from undertaking the optimal strategy. 

The goal of this thesis is to incorporate key features of economical thinking about project valuation in the framework of SDT. The main focus in placed on the approach to risk, time value of money and enabling for multiple sources of risk in the models. 

The first dimension of risk - uncertainty about the monetary outcome - in the economical theory is being in SDT addressed by the utility theory. The second dimension - different valuation of assets based on their correlation to the overall market - has no clear analogy with the SDT theory, which is solved by making the utility function two-dimensional, or in a case of small number of different correlations, multiple utility functions with subscripts. 

The time value of money exists in SDT in a simple form of discounting, which is very similar to discounting with either risk-free discount rate, or risk-adjusted discount rate in economics. In this thesis we present the idea of people being logically indifferent to some sums of money now and in the future  as a function of the ability to borrow, their assets and the ability to invest with no risk.  

In addition to being able to interpret the classical valuation techniques in the form of SDT, the new approach allows us to model any number of different uncertainties by any random process (the economical theory exclusively uses normal and log-normal processes). 

Because the project is modeled by the SDT framework, its value is assessed through the process of dynamic programming - in a case of simple models - or by the advanced theory of approximate dynamic programming (ADP) - in case of more complicated models. 

\bigbreak

The advantages of understanding a project's valuation problem as a stochastic decision problem are demonstrated on two examples of business areas with multiple sources of production. Examples of such business are producers of electric power, water, raw materials that can be extracted by multiple viable methods, public transportation companies and others. 

This class of businesses is chosen due to the variability in available production units of a given material/service and partially due to the some level of author's expertise in the field of power generation. 

\bigbreak
\footnote{This should maybe be rewritten, expressing what I want to achieve, rather than what the new formulation does... }
First, it is shown that the formulation of ROA from authors like Guthrie \cite{Gut:09}, Vollert \cite{Vol:03} and others \cite{} can be interpreted by SDT rather easily, smoothly allowing for multiple sources and models of uncertainty, Bayesian learning (?) and the ability to cope with high dimensional problems. 

Second, a possible generalization of the example is shown, presenting the full power of the new approach. It is expected that by lowering the assumptions of ROA and similar techniques, considering various aspects of project's features with less approximation, a more precise valuation can be achieved. 

The advantage of SDT formulation in comparison to ROA and other models is not as clear-cut as the ROA vs NPV for example, where NPV serves as a lower bound to ROA valuation. The SDT approach tries to incorporate as much information as possible to get the most precise current value of a project, which does not have to be same or higher as ROA. 

The new SDT approach has a goal to incorporate all the standard economical theory, approach to risk, time value of money, different valuation of returns on market-correlated assets, while enabling more complex modeling of uncertainty, Bayesian learning and a way to address projects, that translate to high-dimensional decision problems. 

\textbf{Suggestions: ROA history, more citations, talking about replicating portfolios, the exact structure given by the supervisor}



%\addcontentsline{toc}{chapter}{Introduction}

%\addcontentsline{toc}{chapter}{Preliminaries}

\pagestyle{headings}




\chapter{Preliminaries}
\ruletext{Introduction to preliminaries}

To properly understand a mathematical text it is important to first define the used notions and symbolism. Since this thesis is based on many different authors, from both financial and mathematical world, a short unifying overview of the used theory is important. 

The notation used in this thesis comes predominantly from the most influential authors in the respective fields of study: 
\begin{itemize}
	\item general economy \cite{BerDeM:09};
	\item real options \cite{Gut:09};
	\item stochastic decision theory \cite{BacChi:19}.
\end{itemize}

 The pure mathematical symbolism comes from the author's studying experience at FNSPE CTU and its applicability is proven in his previous works \cite{Rol:18} and \cite{Rol:19}. 
\footnote{Filip == Author or Filip == I/Me} 

\ruletext{}
\section{Used mathematical symbolism}
\ruletext{Sets, random variables, what do I mean by 'probability'}

In the whole thesis, bold capital letters, such as $\mathbf{X}$, represent a set of all elements $x \in \mathbf{X}$ as in \cite{Rol:18}. The cardinality of a set $\mathbf{X}$ is denoted with two vertical lines as $|\mathbf{X}|$. Random variables, understood in a sense of the standard Kolmogorov's probability theory \cite{Kol:60}\footnote{Does this citation make sense? }, are represented with a tilde above the variable, i.e. $\tilde{x}$. Realizations of random variables are denoted by the same letters as the random variable without the tilde, i.e. $x$. 

\begin{definition}(Probability)
	Let $\tilde{x}$ be a random discrete variable. Then $P(x)$ denotes a probability that the realization of $\tilde{x}=x$. Similarly if $\tilde{x}$ is a continuous random variable, then $p(x)$ denotes a probability density of the realization $\tilde{x}=x$. 
\end{definition}
\begin{remark}
	To rigorously unify the notation and simplify the formulas a Radon-Nikodým (RN) density \cite{Rao:87a} is introduced  with the notation $p(x)$ and the name ``probability density``. The dominating measure of this RN density is either the counting measure (in discrete case) or a the Lebesgue measure (in continuous case). The notation $P(X)$ is reserved only for the cases when the discreteness of the argument needs to be emphasized. 
\end{remark}

The last general definition is the definition of well known concept of conditional probability \cite{Jay:03}. 
\begin{definition}(Conditional probability)
	Let, depending on the context, symbol $p(x|y)$ represents either the conditional probability on discrete variables or the conditional probability density on continuous variables. Then the $p(x|y)$ is defined as:
	\begin{equation}\label{eq:condP}
	p(x|y)=\frac{p(x,y)}{p(y)},
	\end{equation} where $p(x,y)$ is a joint probability density of $x$ and $y$. 
\end{definition}

\begin{remark}
	The definition of conditional probability expressed by the equation (\ref{eq:condP}) corresponds with the classic definitions of the conditional probability and conditional probability density in both the discrete and continuous case. 
\end{remark}

<Probably some other definitions that will be needed in the following chapters> 

\ruletext{}

\section{General economics}
\ruletext{Introduction and basic financial concepts. }

This thesis is built on two main theoretical pillars, the theory of corporate finance \cite{BerDeM:09} and stochastic decision theory (SDT) \cite{BacChi:19}. A basic review of corporate finance terminology and procedures is presented in this section with a focus on project valuation techniques. 


\begin{definition}[Project]
	A project is defined as a piece of planned work or an activity that is finished over a period of time and intended to achieve a particular purpose, mainly a wealth increase of a company or an individual. \footnote{First part comes from Cambridge dictionary. Is that ok just to cite it? }
\end{definition}

\begin{definition}[Value]
	The amount of money that can be received for something. \footnote{Again cambridge dictionary...} \footnote{This definition is added to say that by value in economics we mean money that we can obtain from the asset.}
\end{definition}

A value of a project is naturally a function of realized elementary monetary transactions within the project and the potential selling price of remaining assets. To track and model each transaction of a project in detail is in principle possible, but such approach would be way too complex for practical usage. Furthermore, its benefits would most likely not be significant enough to defend the extra effort of decision makers. 


For purposes of project valuation an aggregation of elementary monetary transactions - a concept of cash flow (CF) and free cash flow (FCF) are used in the world of corporate finance. 

\begin{definition}(Cash flow)
	Cash flow is the net amount of cash and cash-equivalents being transferred into and out of a business (project).  \footnote{Investopedia, economical books do not define it. }
	
\end{definition}

\begin{definition}(Free cash flow)
	The incremental effect of a project on the firm’s available cash is the project’s free cash flow \cite{BerDeM:09}: 
	\begin{equation}
		FCF = OCF - Capital Expenditures, 
	\end{equation}
	where $OCF$ is the operating cash flow and $CE$ are the capital expenditures. 
\end{definition}


Cash flows are in their detail nature discrete, each transaction within the project changes the global cash flow. The moments in which transactions legally take place could be taken as individual time instants. 

However, it is easy to imagine understanding the cash flow as a continuous stream of money per time. By the nature of corporate management, one could also expect, that in majority of non-extreme applications, this would not result in a major distortion of cash flow reality. \footnote{Should I discuss here more. I mean that corporate management does not care about each transaction in each store, it cares about daily or even monthly revenues.}

All of this is true also for free cash flow. 

In this thesis we will further work only with the FCF, so from now on, the term \textit{cash flow} will only mean free cash flow. \footnote{Maybe confusing, consider to define only FCF and call it cash flow...}


\ruletext{}

\subsection{Standard valuation metrics}
\ruletext{NPV, DTA, IRR, WACC and others}

Cash flows capture information about value added to a project in given periods. Due to time value of money and different lifespans of projects, one has to come up with algorithms for their consistent and systematic valuation, according to their cash flows. 

Valuation techniques are attempts to aggregate cash flow vectors in one meaningful number to enable decision makers to choose the best investment. 

\paragraph{Net resent value}
First valuation technique, net present value, is arguably the most used valuation technique in capital budgeting \cite{} \footnote{Is this a strong enough statement that I need to cite somebody?}. Its computation is simple and it can be described as a sum of cash flows discounted for the time value of money:

\begin{equation}
NPV = \sum_{t \in \mathbf{T}} \frac{C_t}{r^t}, 
\end{equation}
where $\mathbf{T} = \{0,1,...,|\mathbf{T}|\}$ is a set of time periods in which the cash flows $C_t$ are obtained. These periods are usually years or months, but they can effectively have any granularity the decision maker wants them to. Discounting factor $r$ expresses the time value of money and is usually derived from the current risk-free interest rate given by the central bank of a nation. \footnote{The discussion about negative interest rates and thus r<1  and also the correct discounting rate is left to my valuation in chapter 3.}

The NPV valuation technique is simple to use, assuming we know the discount rate, which is  constant through the project's duration, and free cash flows, that are assumed to be certain. \footnote{Should I put an example here or would that be too trivial? Cash flow is [-400, 100,100,200,200], interest rate 5\% -> NPV is... }

A more advanced approach that acknowledges the variability in both cash flows and risk-free interest rate is called expected NPV(ENPV) :\footnote{Seems like literature does not use this and I made it up. In the eyes of economics, the cash flows seem to be expected values all the times, the distribution is not considered. } \footnote{Also there is ENPV as "Expanded NPV" NPV+options value used by Vollert, Pindyck and others. }
\begin{equation}\label{eq:NPV}
	ENPV = E\left[\sum_{t \in \mathbf{T}} \frac{\tilde{C_t}}{\tilde{r_t}^t} \right],
\end{equation}
where both cash flow and interest rate distributions are  expected to be known. 

\paragraph{Decision tree analysis}
The simplest valuation technique that acknowledges the importance of further management of investments is called decision tree analysis (DTA) \cite{Vol:03}. In addition to time value discounting of cash flows it offers a framework that can incorporate active management of a project, potentially increasing its overall profits. 

The ability to make actions in projects is sometimes interpreted as having \textit{Real Options} \cite {Gue:17} or \cite{Vol:03}.\footnote{Check if Vollert uses ROA as a lens only...  } This confusing terminology might result in misunderstandings. That is why it needs to be emphasized that, in this thesis, valuation that recognizes the ability of a manager to act but ignores the \textit{law of one price} will be called DTA. 


\begin{definition}[Law of one price]
		<Definition of Law of one price> 
\end{definition}

The decision tree analysis is usually used only for valuation of projects with very simple scenarios. However, its structure could be potentially used for much more complex problems. 

The criticism of DTA coming from Vollert \cite{Vol:03} and others \cite{} is that DTA uses a single discount rate for different branches of the project. This is contradictory to the standard rule in the economics that riskier projects should be discounted more \cite{}. 

However, this criticism could be countered with upgraded DTA in which one would allow variable discount rates in different branches rather easily. Furthermore, both Vollert \cite{Vol:03} and  ... \cite{} do not address the problem of the constant discounting rates and use them in their final valuation algorithms as well.  \footnote{Needs to be confirmed}

The discussion about risk and its role in a valuation algorithm is deferred to the section \ref{}. 

\paragraph{Internal rate of return}
The internal rate of return serves usually as a basic threshold for enterprises entering new projects. Each company would have their own internal threshold of IRR, based on their confidence in ability to make more or less returns on their investments. A startup would most likely have an IRR higher than a long time established bank. 

\begin{definition}[IRR]
	The internal rate of return is the interest rate that sets the net present value of the cash flows equal to zero \cite{BerDeM:09}. This means that IRR for cash flows $C_t$ needs to satisfy the following equation:
	\begin{equation}\label{eq:IRR}
		0=\sum_{t=0}^{T}\frac{C_t}{(1+IRR)^t}
	\end{equation}
\end{definition}

The problems with IRR are that additional assumptions on cash flow vector have to be satisfied so that there exists only one unambiguous result of the equation \ref{eq:IRR}. \footnote{Or maybe there are no clear assumptions, but I know that the result of the equation might not exist or it might have multiple results.}

\paragraph{Weighted average cost of capital}

New projects can be financed from two sources, from the company's free capital obtained from other ventures or founders, or in majority of the cases (?) by borrowing money on a market. 

There are two standard alternatives when it comes to borrowing money. The company can either issue stocks and allow investors to participate on its future profits or the money can be borrowed, as individuals do, in a bank. The effective interest rate of the first option is the expected return investors want for their participation in the ventures of given company, while the interest rate of the second option is a pre-arranged contractually supported rate. 

The cost of capital, the effective interest rate of the given mixture of funding for a new project is then given by the weighted average of these two rates, adjusted for a tax benefit in borrowing from a creditor. This rate is called WACC - weighted average cost of capital and is defined as: 

\begin{equation}
	WACC = \frac{E}{E+D}\cdot r_E + \frac{D}{E+D}\cdot r_D (1-\tau_C), 
\end{equation}
where $E$ is the value of equity, $D$ is the value of debt, $r_E$ the equity cost of capital, $r_D$ the debt cost of capital and $\tau_C$ is the corporate tax rate.


\section{Real option analysis}\label{sec:ROA}

\ruletext{What is ROA, introduction and hint of larger future discussion in the next chapter}
	
	
An option in financial world means having a right to buy (or sell) an asset in future for a fixed price (strike price) \cite{BerDeM:09}. Option trading has origins in commodity markets (for example corn or oil), where participants want to in some sense insure themselves against the negative movement of a price on the market. Options that are being traded today on the derivative market span almost every tradeable asset that can be though of \footnote{Find some citation or do not use this sentence.}. 

The value of an option naturally depends on its time to maturity, current price of the asset and a strike price. A proper valuation method for European options with no dividends came with the Black-Scholes-Merton model \cite{BlaSch:73}, which in addition requires only the volatility of the underlying asset and the following assumptions: 
\begin{itemize}
	\item Effective markets ...
	\item Log-normal distribution of the asset price
	\item There are no transaction costs in buying the option
	\item The risk free rate is known and constant. 
\end{itemize}


This established and well received technique for financial option valuation spawned the idea of real option analysis. The option to buy an asset for a given price is similar to the ability to buy an expected future cash flow, i.e. invest in a project. 

<Origins of real options> The first economist, pioneer of the term \textit{Real option analysis}, ... ... treats investments as a complete analogy of trading with options. To be able to delay an investment has a value and..... 

The usage of the phrase real option analysis in literature is rather fuzzy. It is used by many authors such as \cite{BerDeM:09}, \cite{Gut:09} and \cite{Gue:17}, however their usage of the term differs. Three different usage classes of the term real option analysis were identified by the author. They differ by the level of analogy to the valuation of financial options. 

\paragraph{Complete analogy}
First, there is a class of complete analogy. All five parameters of financial options needed for their valuation with BSM model are identified with the parameters of an investment opportunity. An example of a valuation of a car dealership can be found in \cite{BerDeM:09}\footnote{Should I reference page numbers? Will anybody actually look up the example? } where the following identification is made:

\begin{table}[H]
	\begin{footnotesize}
		
		\centering
		\renewcommand{\arraystretch}{1,2}
		\label{Tab:BSModel}
			\begin{tabular}{|l |r|}
			\hline	
			Financial option& Real option \\ \hline
			Stock price& Current market value of asset \\ \hline
			Strike price& Upfront investment required	\\ \hline
			Expiration date& Final decision date \\ \hline
			Risk-free rate& Risk-free rate\\ \hline
			Volatility of stock & Volatility of asset value \\ \hline
			Dividend & FCF lost from delay \\ \hline
			
		\end{tabular}
		\caption{Identification of parameters for real options with respect to the financial option \cite{BerDeM:09}. }
	\end{footnotesize}
\end{table}

Another example can be found in \cite{Que:10} where a telecommunication company is being valued by the complete analogy valuation technique. 

This class of authors focuses on the clear analogy and thus the acknowledged scope of possible manager's actions is limited basically only to timing options. The only decision is to invest to a project now or later.


\paragraph{Partial analogy}

The second class of authors uses only the core property of the financial analogy and that is the law-of-one-price. With the help of this assumption, the authors, eg. \cite{Gut:09} usually derive risk-neutral probabilities which are then used for modeling of some internal variable of the cash flow functions. 

This class of authors is the most numerous and most mathematically rigorous. The core of publications in this class is usually in solving stochastic differential equations, eg. \cite{Vol:03} whereas the role of the assumption about law-of-one-price is used mainly as the ground for obtaining one of the missing parameters of the stochastic model. 

\paragraph{No analogy}

Third class of authors does not use the law-of-one-price and is thus the farthest away from the original idea of ....\footnote{The pioneer name}. This class of authors, i.e. \cite{Kas:04} and \cite{Gue:17}, understands the term real option analysis as a useful lens for looking at the project valuation. They accentuate the value of further managerial decision, but the valuation structure and algorithms do not differ from the DTA approach as defined earlier. 

Thus, this class of authors is declared as a misuse of terminology and not further considered. 

\bigskip

In the next part of the thesis, namely \ref{} the core message behind the term real option analysis will be thoroughly discussed. 

\ruletext{}

\section{Statistical decision theory}
\ruletext{Standard SDT as a framework, states, actions, inputs, outputs, rewards, probability distributions}

The second pillar upon which this thesis stands is the statistical decision theory (SDT). An area of applied mathematics that formalizes and studies optimal decision making of agents. As decision making in its broadest sense encapsulates a vast amount of human behavior, the class of problems it is able to solve is quite large. 

The SDT's main focus is to determine the optimal strategy (a sequence of decisions) to act upon, generally in dynamic and uncertain environment. A classical structure of a decision making problem consists of five building blocks

\begin{itemize}
	\item Set of time epochs - $\mathbf{T}$;
	\item Set of environment states in those epochs - $\mathbf{S}$;\footnote{Possibly different $S_t$ in different times.}
	\item Set of actions in those states - $\mathbf{A}$;
	\item Reward function of transition from one state to another - $r(s_t|a_t,s_{t-1})$;
	\item Transition probabilities governing the transitions from one state to another $p(s_t|a_t,s_{t-1})$.
\end{itemize}

The set of time epochs, states, actions is usually known, defined by the structure of the decision problem that is being solved. Reward and transition functions tend to be unknown in solving these problems and they need to be often somehow estimated. 

Usually, the biggest task in SDT is to correctly approach the uncertainty about transition probabilities between the different states of a project. There are two approaches to parameter estimation in statistics, classical approach and a Bayesian approach. Since the Bayesian approach seems to fit the format of decision making better - allowing for smooth updating on newly observed data - it is used in this thesis. 

The goal of SDT is to find the optimal strategy - sequence of actions. The optimality of such strategy is defined as it having the maximal expected cumulative reward among all eligible strategies 

\begin{equation}
	\pi^*=\argmax_{\pi \in \mathbf{\Pi}} E\left[\sum_{t\in \mathbf{T}} r(s_t|a_t,s_{t-1})|\pi\right].
\end{equation}

This maximization can be in total absolute values (in finite or discounted cases) or per time period (mostly in infinite non-discount cases). Due to the economic nature of this thesis we will focus on the total cumulative reward of a finite process (?). 

\subsection{Dynamic programming}
To maximize over all possible strategies by computing the expected cumulative reward for each one of them is a very demanding task even for low-dimensional decision problems. 

Thus, a clever idea of backward induction called dynamic programming is used. A function, called the value function is defined on the set of all possible states $\mathbf{S}$. This function represents the expected cumulative reward to be obtained from the given state onwards. The idea of backward induction is based on the truth that a sequence of actions is optimal if and only if the last action is optimal. 

This clever computation of value functions from the problem horizon backwards through all the possible states of the problem decreases the complexity from exponential to polynomial. Instead of maximizing over $|\mathbf{A}|^{|\mathbf{S|^{\mathbf{|T|}}}}$ possible strategies at once, one needs to compute significantly less demanding complexity of $|\mathbf{A}|\cdot|\mathbf{T}|\cdot{|\mathbf{S}|}$. \footnote{Check this}

The formula representing the backward induction is called the Bellman equation: 

\begin{equation}\label{eq:Bellman}
	V(s_{t-1}) = \sum_{s_t \in \mathbf{S_t}} p(s_t|a_t, s_{t-1}) [r(s_t|a_t, s_{t-1})+V(s_t)].
\end{equation}

By defining the value function on the horizon, we can compute value functions of states with lower and lower time indexes, until we get to the time 0, which represents the present. Not only that we have the expected value of the optimal decision making, but we have also derived the optimal strategy for every possible path through the state space. 

The backward induction reduces the computation complexity significantly. However, for even a moderate-dimensional decision problems, the number of computations is still extremely large. 

The problem of computational complexity of dynamic programming is called "three curses of dimensionality" \cite{Pow:11} and various solutions have been proposed. These solutions are as a group referenced as approximate dynamic programming. 


\subsection{Approximate dynamic programming}

The computational complexity of dynamic programming for moderate and high-dimensional decision making problems is so demanding that results cannot be obtained in a reasonable amount of time. 

The response to this problem comes in a form of approximate dynamic programming, a section of decision making under uncertainty, that is represented by a number of algorithms that are trying to obtain quasi-optimal strategies with more reasonable demand for computation power. 

There are many different algorithms, that try to obtain approximate results of the precise dynamic programming represented by the bellman equation. In this thesis the ADP algorithm called <Q-learning, SARSA...> is used because of its high performance in ..., while being still relatively easy to implement. A longer discussion of its choice is left to its corresponding chapter \ref{}. 

\paragraph{<Q-Learning, SARSA,..>}
<Detailed description of the chosen ADP algorithm> 


\subsection{Bayesian statistics}

The field of mathematical statistics can be divided into two branches, classical (also called frequentist) and Bayesian. The philosophies of each one are fundamentally different, however in principle, they can serve for revealing new truths of the measured data in a similar fashion. 

Mathematical statistics is a very broad topic, not possible to summarize it in one paragraph. The use of Bayesian statistics in this thesis is only as a tool, no broader discussions about the internal philosophy of different approaches are presented.

In general, statistical theory is used to determine a distribution from which the observed data come from. In majority of cases, it is assumed that the data are realizations of a random variable with a distribution from some parameterized class - normal, log-normal, poisson, etc. The goal is then to determine, with some level of confidence, the parameters that fit the observed data in some sense the best. \footnote{Large simplification, statistics can be used in many different ways.} 

The main difference between the Bayesian and classical statistics is how the parameters of a distribution are perceived by the statistician. In the classical theory, it is assumed that observed data come from some distribution with some firm but unknown parameters $\Theta$. In contrast, the Bayesian view on the parameters is such that they are perceived as random variables $\tilde{\Theta}$. 

This terminology twist can be a source of initial confusion for frequentist statisticians, but it allows a simple and elegant update of parameter estimates with the Bayes formula.

\begin{equation}
	p(\Theta|d)=\frac{p(d|\Theta)p(\Theta)}{p(d)}, 
\end{equation}
where $\Theta$ is generally a multivariate parameter and $d$ are observed data. \footnote{The p(d) in denominator needs to be rewritten as integral if this formula is really to be used.}

The interpretation of Bayes formula, is that the distribution of parameter $p(\Theta)$ called the prior distribution, is updated for the newly observed data $d$, providing new, posterior,  distribution $p(\Theta|d)$. 

This update can be understood as learning about the "true value" of a parameter, which is very useful structure for dynamic decision problems. 

Since the Bayesian theory tells us only how to update an already existing distribution, a prior distribution needs to be given, even though no data were measured yet. 

This problem is in Bayesian statistics understood as an advantage, since one can use his knowledge about the problem that is being solved and incorporate it to the prior distribution, which is then updated on the measured data. 

The task of consistent creation of prior distribution is a complicated topic and can be found in more detail in \cite{Ber:85}. Furthermore the prior information always exists, as Peterka \cite{Pet:81} puts it: "No prior information is a fallacy: an ignorant has no problems to solve".  


\subsection{Utility}
The concept of utility instead of monetary or other globally measurable gain comes in when the gains are valued non-linearly. 

 Multiple studies show \footnote{Find citations (?) or omit this formulation}, that the majority of people are risk-averse, meaning that the value of uncertain monetary gain is not equal to its expected value. 

One of the simplest example to demonstrate the usage of utility is given by \cite{BacChi:19}. Imagine an individual is given a choice, either to get 500\$ right away or to gamble for 1000\$ in a fair coin toss. A rational decision maker driven only by the expected value of his actions would be indifferent to the two choices. However, the majority of people tend to take the certain amount instead of gambling. 

This example can be reformulated as follows: How much money would the decision maker need to obtain for certain so that he would be indifferent to gamble for a 1000\$. In other words, how much the risk-averse person values that gamble. 

The non-linearity of utility obtained from large amounts of money is only more understandable for very large sums of money. There is a little difference for an average human in obtaining 10M USD and 20M USD. The change in the person's life will be almost the same and presumably positive. However one result is certain and the other one has only a probability of 1/2. 

Another interesting example of the risk-aversion of people is the famous St. Petersburg paradox first formulated by Bernoulli in 1738, \cite{Ber:54}. A risk-neutral \footnote{Define risk-neutral (?)} decision maker would be willing to pay any amount of money to be able to play a game defined by the paradox. However it is shown that people seldom value the game more than 25 USD, which corresponds to a case that the initiator of the bet does not have an infinite amount of money, rather only 16,5M USD \cite{}. \footnote{This is from wikipedia, find more cool sources. Interesting, but does not have to be in the thesis}

Regarding to utility there is also an interesting asymmetry in human psychology about obtaining gains and incurring losses. The graphical expression of this asymmetry can be found in \cite{BacChi:19}. \footnote{Put the picture here, or cite the exact page?} 

The utility function of each decision maker is different and an approximation of its shape can be obtained by an algorithm based on a questionnaire, which also ensures the consistency of responses of a given individual. 


\chapter{Project valuation as stochastic decision problem}

<Make an intro to valuation by SDT, why did I choose the layout I did, what I want to achieve? \footnote{Will be known after a meeting.}> 

\section{Simple timing option}

First, the valuation of project with the SDT framework will be demonstrated on a very simple example. We focus on a project that consist of an opportunity to build a production facility that transforms one raw input into one product. This opportunity is expected to be temporary and vanishes in a predefined time epoch. The facility has a known fixed lifespan (not dependent on its building time) after which its salvage value is 0. 

For simplicity only one active managerial action is considered, the option to build the facility. Building the facility takes one time epoch and it has 100\% success rate. The two passive managerial actions do not influence the state of the project. The first passive action is to wait for the price movement of the next epoch and the second is production of products in the facility after it is built. In this example, after building the facility there is no way of stopping the production, no stopping, mothballing of switching options are considered. 

Two variables of the project are considered uncertain. These are the price of the input and the price of the final product. Current prices are expected to be known and the future prices are modeled by the binomial model. 

The transition probability function values that we use in our computations are expected to come from the ROA as risk-neutral probabilities. 

The time value of money is in this simple case solved by discounting the future money by known risk-free rate. The risk-aversion of investors is not accounted for in this section.

To value this simple class of projects via the SDT framework the 5 SDT sets need to be defined. 

\paragraph{Time set}
The set of time epochs is defined as a sequence of natural numbers starting with 0 - the present. The actual time difference between the time epochs is constant and it can theoretically be any time interval imaginable. For application purposes, this interval is usually month, three moths or a year. The final time epoch of the set is expected to be the time, where project ends and the facility (if existing) is sold for its salvage value. The time epoch set can be expressed mathematically as: 

\begin{equation}
	\mathbf{T} = \{0,1,2,...,|\mathbf{T}|-1\},  |\mathbf{T}| \in \mathbb{N}
\end{equation}

\paragraph{State set} 
The set of states is represented as an ordered vector of three numbers $(b,x,y)$ in which the observed reality of project's environment is encoded. The first number $b$ represents if the facility is build and working ($b$=1) or not ($b$=0). The numbers $x$ and $y$, in the style of Guthrie \cite{Gut:09}, represent the number of times the price of input (or product) has risen. Together with the information about time epoch of a state, the price can be determined as: 

\begin{equation}
	p_t = p_{0} U_x^{2t-x},
\end{equation}
where $p_0$ is the initial price and $U_x$ is the growth coefficient for input price in this case. 


Because of the binomial structure of the price modeling, the set of achievable states is different for each time epoch. In the first epoch, there is no possibility of price movement up or down and the facility is surely not built yet. This implies $\mathbf{S_0}=\{(0,0,0)\}$. All the future state sets have cardinality higher than 1 and can be generally described as:

\begin{equation}
	\mathbf{S_t} = \{(b,x,y,)|b \in \{0,1\}, x \in \{0,1,2,...,t\}, y \in \{0,1,2,...,t\}\}
\end{equation}

\paragraph{Action set}
The set of possible managerial actions is rather simple. The project's manager is able to build the facility when it has not been built yet, and the time option did not expired yet. In mathematical terms: 

\begin{equation}
\mathbf{A_t(s_t)}=
\begin{cases}
\{0,1\} & \text{if}\ s_t=(0,x,y) \land t < t_{exp}  \\
\{0\} & \text{otherwise},
\end{cases}
\end{equation}

where $t_{exp}$ is time of building option expiration. The action $a_t = 0$ is called either "waiting" if $s_t = (0,x,y)$ or "producing" if $s_t = (1,x,y)$. By the structure of the project it is clear that there can be only one action $a_t=1$ through the lifespan of a project of this type. 

\paragraph{Transition probabilities}
The transition probability function is constructed as a product of transition probabilities between each dimension of the state vector. The transition probability function in the first dimension can be described with the Kronecker delta function, since the state of a facility is changed deterministically based on manager's action. The probabilities of change in other two dimensions are modeled by the risk-neutral probabilities of up movements $p_{Ux}$ and $p_{Uy}$ and their complements $p_{Dx}=1-p_{Ux}$, $p_{Dy}=1-p_{Uy}$, which represent down movements. Overall the transition probabilities between states have four distinct values and can be defined as: 

\begin{equation}
	p(s_t,a_t,s_{t+1})=p(b_t,b_{t+1},a_t)p(x_t,x_{t+1})p(y_t,y_{t+1}), 
\end{equation} 
where
\begin{align}
	p(b_t,b_{t+1},a_t) &= \delta_{b_{t+1},a_t} \label{eq:build}\\
	p(x_t,x_{t+1}) &= \delta_{x_t,x_{t+1}}*P_{Dx} +\delta_{x_t,x_{t+1}-1}*P_{Ux} \\
	p(y_t,y_{t+1}) &= \delta_{y_t,y_{t+1}}*P_{Dy} +\delta_{y_t,y_{t+1}-1}*P_{Uy}, 
\end{align}

where in the equation \ref{eq:build} we need to remember that this is true only for the allowed actions $a_t \in \mathbf{A_t(s_t)} $.

\paragraph{Reward function}
The reward function has two main elements. First there is a large negative reward (cost) for the action of building, lets call it $C$. The second type of reward is the profit that a running facility is making. This profit is simply modeled as a price of product minus the price of input. We expect that one unit of input creates exactly one unit of output and that no other costs occur in the process. This is a strong simplification, but it is adequate for our first example. The reward function can be mathematically described as: 

\begin{equation}
	r(s_t,a_t,s_{t+1})= \delta_{1, a_t}C + \delta_{b_{t+1},1} \left(p_{x0}*U_{x}^{2t+2-x} - p_{y0}*U_{y}^{2t+2-y}\right),
\end{equation}
where $p_{x0}$ and $p_{y0}$ are the current prices of input and product respectively. 

\subsection{Solving the valuation task}
Now that the structure of a simple project is defined, its actual valuation can be computed. Since the structure prepared a solution via dynamic programming, this is used for the actual valuation. 

The dynamic programming algorithm computes the value function from the states at the horizon up to the first one by Bellman equation, \ref{eq:Bellman}, \footnote{Add discounting in bellman in preliminaries} where the future cash flow is discounted by the risk-free discount rate.

The valuation of a project is obtained as the discounted expected reward. The algorithm of dynamic programming further yields the optimal strategy. In this simple example, the optimal strategy says when, and even if, to build the facility based on the price evolution of input and product. 


\section{General options}
In the previous chapter only an example of simple timing option was introduced. Now we want to focus on projects which have general options, with possibly complex structures and dependencies on each other. Guthrie \cite{Gut:09} and others \cite{Vol:03} offers a well elaborated lists of possible types of options. We consider the following to be covering the absolute majority of possible real options in real applications: 

\begin{itemize}
	\item Timing options (simple and compound)
	\item Scale options
	\item Switching options
	\item Abandoning/mothballing options 
	\item Learning options
	\item Expansion/Contraction options
	\item M\&A options
	\item joint venture options
	\item ...
\end{itemize}

All of these options represent different actions of the management trying to alter (or create) the production process of a project. Every action that these options cover results in a change of parameters or shape of reward function. 

\paragraph{Single timing options}
Single timing options, introduced in previous section, consist of an option to delay the investment in a facility. They have two implications for the reward function. First there is (usually) large negative reward representing the cost of the facility and second is the (hopefully) positive reward from transformation of inputs into products when the facility is built. 

A generalization of the timing option in the previous chapter can be done in the following ways: 

\begin{itemize}
	\item The time of building is not one time epoch and its success is not 100\%
	\item The cost is not clear, it is described as probability distribution
	\item The time of building is also described as a distribution. 
	\item .... 
\end{itemize}

\paragraph{Compound timing options}
Compound timing options allow to understand building a facility as a sequence of steps. This thinking about a project allows an early abandonment in case of bad evolution in the environment. A clear sequence of steps is predefined, following steps can be ordered to be built after the previous steps are already finished and the production does not start until all steps are successfully done. 

In terms of SDT, the possible action set is a simple generalization of the action set presented in the primary example: 

\begin{equation}
\mathbf{A_t(s_t)}=
\begin{cases}
\{0,1\} & \text{if}\ s_t=(0,x,y) \land t < t_{exp1}  \\
\{0,2\} & \text{if}\ s_t=(1,x,y) \land t < t_{exp2}  \\
... \\
\{0,\mathbb{S}\} & \text{if}\ s_t=(\mathbb{S}-1,x,y) \land t < t_{exp\mathbb{S}}  \\
\{0\} & \text{otherwise},
\end{cases}
\end{equation}
where $\mathbb{S}$ is the final building stage of the facility. 

\paragraph{Scale options}
<Here I would talk in detail about scale options>

\paragraph{Switching options}
<Here I would talk in detail about switching options>

\paragraph{...}

\section{ADP approach}
<I have not studied yet different approaches thoroughly. I believe that the optimal ADP approach depends on what actually the ROA in SDT will be.> d


\section{Approach to risk}
<I have an idea about how to approach risk with risk-averse investors (which are risk-averse for 1. actual variation 2. correlation to the overall market), but it has a problem. It is not consistent when the time intervals are being shortened...> 






\chapter{Valuation of projects in multiple-source industries}
<Explain the class, say what are the examples, what uncertainty these industries need to cope with, what are the options? > 

The class of .... valuation problems is ideal for demonstration of the usability of the newly developed valuation technique. Due to the inherit uncertainty in this field and many actions that can be undertaken, the additional value  assigned to the investment opportunity can be significant. 

We start with a rigorous mathematical definition of this class of valuation techniques in terms of SDT. Then we pick one example and promptly show that the number of possible states is exponential with respect to... Approximate dynamic programming techniques like Q-learning or SARSA were developed in order to solve exactly these types of problems. Due to their strengths and only a minor flaws their usage is justified. 

\section{Power company}
<Complete valuation of new power plant project by the new technique>

<Complete valuation of new power plant project by ROA>  

\section{Public transportation/Water treatment/}
<Complete valuation of ... by the new technique>

<Complete valuation of ... by ROA>  

\chapter{Discussion}
The new approach is better because it solves the current problems with ... Also the applicability of the new approach is in my opinion broader since it can address multiple sources of uncertainty. Furthermore the power of the decision making process is kept in the hands of the decision maker through creation of prior distributions. The manager is guided through the world of utility functions and priors, which both can be created from a set of simple questions about gambles and beliefs of the manager. The creation and usage of the utility and prior density functions are fool-proof in a sense of mathematical coherence. 

\chapter{Conclusions}
In my master's thesis I have rigorously compared the state-of-the-art valuation techniques used in present investment companies. I have shown the advantages and disadvantages of real option analysis  and stochastic decision theory. The combination of these, which I call ...,  yields a new view on the world of risky investments that empowers the decision maker and thus allows for better adoption in the rigid environment of investing. 


The results of the new approach are promising and this thesis could be understood as a first step towards a broader usage of SDT in the field of project valuation. 














\chapter{Inspiration}

\section{Problems of the current approach}
I will list here the problems I think my current approach to project valuation has. 
\begin{itemize}
	\item Bad time scaling. If I will use half year or quarter year time intervals instead of year intervals, the result would be different due to the general shape of the utility function. It seems like the valuation technique is influenced by the time intervals used. 
	\item 
\end{itemize}



The new valuation approach should incorporate the advantages of ROA into the stable framework of SDT, improving the performance of each approach individually, namely it should:

\begin{itemize}
	\item Capture uncertainty of a project
	\item Allow managers to implement their own approach to risk. 
	\item Enable to rigorously handle a time devaluation of money according to the profile of the company making the investment. 
	\item Allow to systematically compare projects in a portfolio to find the best candidates for an actual investment. 
	\item <Add all other qualities the new valuation technique should have> \footnote{For example allow for Bayesian learning, cope with a high-dimensional problems, multiple uncertainties,... }
\end{itemize}



\section{Black-Scholes Merton model}
Only four parameters and one assumption is needed to determine a value of an option according to BSM model for option pricing. Assume that the market is complete, and thus the law of one price holds \cite{}. Then to value a option you need to know only its time to maturity, its strike price, the current price of the underlying stock and its volatility as follows \cite{BerDeM:09}: 
\begin{equation}
C = SN(d_1) - PV(K)N(d_2), 
\label{BSMModelEq}
\end{equation}
where $S$ is the strike price, $PV(K)$ is a price of a bond paying K on the expiration day of the option and $N(d)$ is a cumulative normal distribution, probability that a normally distributed variable is less than $d$. Value of $d_1$ and $d_2$ is then defined as: 
\begin{equation}
d_1 = \frac{ln(S/PV(K))}{\sigma \sqrt{T}}+\frac{\sigma \sqrt{T}}{2}
d_2 = d_1 - \sigma \sqrt{T}
\end{equation}

The dependency of the price of an option is positive in case of volatility and time to maturity Increasing these parameters leads to a higher option price. On the contrary the rise in current stock price or strike price of the options lowers the value of an option. 



\section{Project and cash flow}

\begin{definition}
	A project is defined as a piece of planned work or an activity that is finished over a period of time and intended to achieve a particular purpose, mainly an increase of company's or individual's wealth\footnote{First part comes from Cambridge dictionary. }.
\end{definition}

Examples of a project are: 
\begin{itemize}
	\item developing a cooper mine; 
	\item innovation of chemical processes in an oil refinery;
	\item upgrade of current machinery in a production line; 
	\item changing the form of software development philosophy towards agile practices.
\end{itemize}

When examining the expression max E\{NPV(Options)\} four observations come to mind. 

First, the maximization is over some set of control strategies. This set can be generally large, even uncountable. Due to the class of the problems that are addressed in this thesis, the actions made by a manager are not expected to be continuous, not even very frequent. This would result in an assumption of small control strategy space, at least for now. 


 
 
\section{Bachelor's thesis parts that could be useful for TeX styling}




\begin{algorithm}
	\caption{Finding the optimal policy for a \textit{single system} MDP with known $P$}\label{alg:SingleKnown}
	\begin{algorithmic}[1]
		\Require{$\mathcal{M}=(\textbf{T},\textbf{S},\textbf{A},P,R)$}
		\State$ \varphi_N^{o}(s) \gets 0$, $\forall s \in \mathbf{S} $ \Comment{Based on Definition \ref{ValueF}.}
		\State $t \gets N$ 
		\While{$t\ne 0$}
		\For{ each $s \in \{1,2,...,|\mathbf{S}|\}$}
		\State $\varphi_{t-1}^{o}(s) \gets Equation$ $(\ref{DynProgEq})$ \Comment{With known $P$ and $\varphi_{t}^{o}$}
		\EndFor
		\State $t \gets t-1$
		\EndWhile
		\State $\pi_{0}^{o}(s) \gets argmax$ $\varphi_0^{o}(s)$ $ \forall s\in \mathbf{S}$, \Comment{Deriving the optimal policy}
		\State	\Return{$\pi_{0}^{o}$}
	\end{algorithmic}
\end{algorithm}


\pagestyle{plain}
\bibliographystyle{plain}
\bibliography{fr}


\end{document}
